\documentclass[12pt, a4paper]{article}

\usepackage[left=2cm, right=2cm, top=2cm, bottom=2cm,
paperheight=6in,paperwidth=8.5in]{geometry}

\usepackage{enumitem}
\usepackage{courier}
\usepackage[T1]{fontenc}
\usepackage[T1, hyphens]{url}
\usepackage[polish]{babel}
%\usepackage[utf8]{inputenc}
\usepackage{amssymb, amsmath}
\usepackage{listings}
\usepackage[unicode, pdfdisplaydoctitle]{hyperref}
\usepackage{xcolor}

\hypersetup{
	pdftitle={Sortowanie - dydaktyka},
	pdfsubject={Zadanie 3},
	pdfauthor={Konstanty Dmochowski, Magdalena Jarmużewska},
	pdfkeywords={sortowanie, wykres},
	pdfcreator={},
	pdfproducer={}
} %podstawowe elementy kluczowe dokumentu

\pdfpageattr{/Group << /S /Transparency /I true /CS /DeviceRGB>>} 

\lstset{
	literate=
	{ą}{{\k a}}
	{Ą}{{\k A}}
	{ż}{{\. z}}
	{Ż}{{\. Z}}
	{ź}{{\' z}}
	{Ź}{{\' Z}}
	{ć}{{\' c}}
	{Ć}{{\' C}}
	{ę}{{\k e}}
	{Ę}{{\k E}}
	{ó}{{\' o}}
	{Ó}{{\' O}}
	{ń}{{\' n}}
	{Ń}{{\' N}}
	{ś}{{\' s}}
	{Ś}{{\' S}}
	{ł}{{\l}}
	{Ł}{{\L}}
}

\definecolor{myBlue}{RGB}{55,118,171}
\definecolor{orange}{RGB}{255, 211, 67}

\begin{document}
	\begin{flushleft}
		\Huge{\textbf{\textcolor{myBlue}{P}\textcolor{orange}{y}\textcolor{myBlue}{t}\textcolor{orange}{h}\textcolor{myBlue}{o}\textcolor{orange}{n}}}
	\end{flushleft}
	\begin{center}
		\textbf{SORTOWANIE - DYDAKTYKA} \\[0.5cm]
	\end{center}
	\begin{enumerate}[label=\textbf{\arabic*.}]
		\setcounter{enumi}{2}
		\item W tabeli poni{\. z}ej przedstawiono wspomniane wcze{\' s}niej dwa typy sortowa{\' n}, wraz z okre{\' s}lonym rozmiarem tablicy $n$: 
		\begin{table}[h!]
			\centering
			\begin{tabular}{|c|c|c|c|c|c|c|l|l|}
				\hline
				$\boldsymbol{n}$ &
				\textbf{100} &
				\textbf{1000} &
				\textbf{5000} &
				\textbf{10000} &
				\textbf{50000} &
				\textbf{100000} &
				\textbf{500000} &
				\textbf{1000000} \\ \hline
				\textbf{\begin{tabular}[c]{@{}c@{}}Sortowanie \\ b{\k a}belkowe\end{tabular}} &
				&
				&
				&
				&
				&
				&
				&
				\\ \hline
				\textbf{\begin{tabular}[c]{@{}c@{}}Sortowanie \\ szybkie\end{tabular}} &
				&
				&
				&
				&
				&
				&
				&
				\\ \hline
			\end{tabular}
		\end{table}
		\begin{enumerate}[label=\textbf{\alph*)}]
			\item Uruchom program \texttt{sortowanie.py} (kod {\' z}r{\' o}d{\l}owy mo{\. z}esz znale{\' z}{\' c} na stronie \url{https://github.com/Programtica/Porownanie-algorytmow-sortowania}, b{\k e}dzie on dla Ciebie dost{\k e}pny przez 30 dni). 
			\item Wpisz w odpowiednie pola tej tabeli czasy dzia{\l}ania poszczeg{\' o}lnych algorytmów. Na ich podstawie narysuj wykres na kartezja{\' n}skim uk{\l}adzie wsp{\' o}{\l}rz{\k e}dnych (o{\' s} pozioma - rozmiar tablicy, o{\' s} pionowa - czas).
		\end{enumerate}
		%Jak s{\k a}dzisz, jaki wykres przedstawiaj{\k a} obydwa algorytmy?
		Jak s{\k a}dzisz, jaki typ wykresu opisuje poszczeg{\' o}lny algorytm?
	\end{enumerate}
\end{document}